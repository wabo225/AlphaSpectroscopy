\documentclass[12pt]{article}
\usepackage[margin=0.8in]{geometry}
\usepackage{blindtext}
\usepackage{multicol}

\usepackage{amsmath}

\setlength{\columnsep}{1cm}

\title{Alpha particle spectroscopy}


\author{Will Bodron \\ Department of Physics \& Astronomy \\ University of Kentucky}

\begin{document}

\maketitle

% \begin{abstract}
%     Lorem ipsum dolor sit amet, consectetuer adipiscing elit. Etiam lobortis facilisis sem. Nullam
%     nec mi et neque pharetra sollicitudin. Praesent imperdiet mi nec ante. Donec ullamcorper, 
% \end{abstract}

\begin{multicols}{2}

    \section{Introduction}
    % Scope and significance
    \paragraph{} Monoenergetic alpha particles are emitted via nuclear decay from ${}^{210}Po$ atoms. By varying the density of an air absorber and measuring the energy of $\alpha$-particles incident on a silicon surface detector, we can determine the energy dependence of the rate of energy loss for $\alpha$-particles. \cite{kovash} The `curve' generated by the relationship between peak energy and air density is known as a Bragg curve, and is commonly used medicine to determine (and even leverage) the depth at which energy loss occurs. In radiation therapy for cancer, particular Bragg Peaks are used to minimally invasively target cites underneath tissue by focusing the depth of greatest energy loss on the tumor \cite{brookhaven}. Most Bragg curves spike strongly just before their stopping point. This has the effect of leaving little damage to surrounding healthy tissue. It is therefore imperative to be able to measure accurately the rate of energy loss as a function of the thickness of the absorber.
    \paragraph{} We may also use alpha particle spectroscopy to identify which radionuclide they originated from. ${}^{210}Po$ is especially rare, and determining the isotope you may have created by some nuclear process or found in a naturally occurring sample can be done by examining the resulting energy curves obtained via alpha spectroscopy.

    ${}^{210}Po$ has 84 protons, an atomic mass of $209.98 u$, and a half-life of $138.376 d$ \cite{table}. Polonium-210 almost always performs alpha decay, resulting in ${}^{206}Pb$.
    \begin{equation}
        {}^{210}Po \rightarrow \alpha + {}^{206}Pb
    \end{equation}
    With such a short half-life and predictable nuclear decay pattern, polonium-210 has been studied in depth and has been the subject of controversies such as the assassination of one Alexander V. Litvineko \cite{roessler}.


    \section{Theory}
    
    \paragraph{} We first note that $\alpha$-particles are emitted isotropically from a source \cite{kovash}. Therefore, we determine the solid angle subtended by the detector in order to approximate the total number of particles emitted. Given a distance from the source to the center of a circular detector as $R$ and the diameter of the silicon detector, $d$, we find that the solid angle subtended is:\footnote{A small angle approximation of this solid angle is a bit optimistic as the azimuthal angle to the outer edge of our detector is ~$9.2^\circ$}
    \begin{equation}
        \Omega = 2\pi\left(1-\frac{R}{\sqrt{R^2 + \frac{d^2}{4}}}\right)
        \label{solidAngle}
    \end{equation}
    Then, pumping a vacuum between the source and the detector prevents the alpha particles from being stopped by the air absorber. Thus the total number of decays per second can be written as:
    \begin{equation}
        \frac{\text{decays}}{\text{second}} = \frac{N_{detected}\Omega}{4\pi t_{trial}}
    \end{equation}
    Where $t_{trial}$ represents the duration of the detection period. Curies, then, are defined as $1 Ci = 3.7 \times 10^{10}$ decays/second. We will use the MCA spectrum analyzer and the MAESTRO software to collect a spectrum over a duration of 5 minutes, and integrate over all channels to find the number of alpha particles that landed on the detector. \cite{kovash}

    \paragraph{} MCA channels are analogous to detected particle energy and there exists a linear calibration to convert channels to energy values. This calibration can be created using a model 419 pulser, capable of generating pulses similar to those created by the silicon surface detector. By attenuating the height of these regularly generated pulses, we can perform a linear fit to the attenuation-channel data-points. The x-intercept of this fit defines the pedestal channel which represents detected particles of 0 energy. The fit can then be normalized using a single data point. We know the energy of an alpha particle in a vacuum to be 5.3 MeV. Define the energy of the centroid channel for a spectrum obtained at a minimum pressure to be 5.3 MeV. These two data points can be used to interpolate the relationship between MCA channel and energy. \cite{kovash}

    The theoretical value of the energy loss of a projectile is modelled via the particle's dominant loss mechanism, its interaction with atomic electrons. The Bethe-Bloch formula
    \begin{equation}
        -\frac{dE}{dx} = 2\pi r_e^2(m_ec^2)^2Z_p^2N\left(\frac{m_p}{m_e}\right)\left(\frac{B}{E}\right)
        \label{bethebloch}
    \end{equation}
    with $Z_p$ as the particle charge, $N$ as the target density in atoms, and $B$ as the atomic stopping number, calculated as
    $$B = Z_t\ln\left(\frac{4m_eE}{m_pI} - 0.90\right).$$
    The mean ionization energy, $I$, is 32.5 eV for an air absorber, and the value of $Z_t$ is the weighted mean of the values for each of the constituent atoms.
    In order to find rate of energy loss through a medium, we must fill the space between our source and detector with substance of known areal density. The areal density is defined as
    \begin{equation}
        \tau = \rho \times L
        \label{areal}
    \end{equation}
    where $\rho$ is the volume density and $L$ is the linear thickness, and it can be calculated for air at a known temperature and humidity. We can also convert loss rate to units of thickness as an areal density using 
    \begin{equation}
        \frac{dE}{d\tau} = \frac{1}{\rho}\frac{dE}{dx}
        \label{arealConversion}
    \end{equation}
    This derivate will be measured directly by looking at $\Delta E$ and $\Delta \tau$ for every adjacent pairs of data points.
    Finally, the range of a particle of energy $E_0$ can be calculated from the rate of energy loss using 
    \begin{equation}
        R(E_0) = \int_0^{E_0} \frac{dE}{-dE/d\tau}
        \label{range}
    \end{equation}


    \bibliographystyle{plain}
    \bibliography{refs.bib}
\end{multicols}
\end{document}